\documentclass[11pt,a4paper]{report}
\usepackage{marvosym}
\usepackage{hyperref}

\assignment{4}
\group{1 (Moodle), 39 (INGInious)}
\students{Victor Carballes Cordoba (NOMA : 34472100)}{Krystian Targonski (NOMA : 42942000)}

\begin{document}

\maketitle

\section{Constraint Programming (7 pts)}

\begin{enumerate}
\item Identify the goal of each set of constraints in \texttt{sudoku.py}. You can help yourself by looking 
at \url{https://www.pycsp.org/documentation/constraints/} to find the definition of the different constraints. \textbf{(1 pt)}
\end{enumerate}

\begin{answers}[4cm]
Constraints \#1:  Only one amazon in each row\\
Constraints \#2:  Only one amazon in each column\\
Constraints \#3:  The sum of all truth values in both diagonals is equal to 1\\
Constraints \#4:  The sum of all  truth values in a circle, is between 3.2 and 4.8 units, representing all 3x2 and 4x1 moves (combined with a few of the other constraints, that is why this works).
\end{answers}



\begin{enumerate}
	\setcounter{enumi}{1}
	\item Find at least \textbf{two different sets of variables} to model the N-amazon problem.
	For each, describe the variables that you will use to model the N-amazon problem. 
	For each variable, describe what they represent \textbf{in one sentence}. Give their domains.
	If you use arrays of variables, you can give one explanation for the array as a whole (and not for each of its elements), 
	and you need to give its dimensions.
	Choose the most appropriate set of variables. Justify your choice. \textbf{(2 pt)}

\end{enumerate}

\begin{answers}[6cm]

First set of variables, is the simple NxN grid, it has a domain of {0,1}, and is quite easy to understand how it all operates. But this one involves a lot of checks, but at least the domain is small.
The second representation is harder to understand, and that is to use the array variables, with the index as X, and the value as Y. This means that you can even remove one of the constraints from the previous question. Its dimension is then the range between 0 and N. The final approach is this one.

\end{answers}

\begin{enumerate}
	\setcounter{enumi}{2}
	\item  Give the constraints that you will use to model the N-amazon problem.
	For each constraint, also describe what it enforces.
	Your model must take account of the already placed amazons. \textbf{(2 pt)}

\end{enumerate}

\begin{answers}[10cm]
\begin{enumerate}
    \item The values at index x, must be value y for all placed amazons at position x,y
    \item All the variables must be different (different Y)
    \item For each amazon, its diagonals must only contain one piece, represented by the sum of the difference $\sum_{x_{other}}^N(abs(x_{self} - x_{other}) == abs(amazon_Y[x_{self}] - amazon_Y[x_{other}])) = 1\; \forall x_{self}$
    \item For each amazon, no amazon must be in the circle
    \begin{enumerate}
        \item $\sum_{x_{other}}^N(10 < (x_{self} - x_{other})^2 + (amazon_Y[x_{self}] - amazon_Y[x_{other}])^2 < 18) = 0\; \forall x_{self}$
        \item For implementation reasons, its enfeasable to use float values in the conditions, so after testing a bit, these values worked flawlessly instead of the $3.2^2$ and $4.8^2$ that are harder to make.
    \end{enumerate}
\end{enumerate}
\end{answers}

\begin{enumerate}
	\setcounter{enumi}{3}
	\item Modify the \texttt{amazons\_cp.py} file to implement your model.
	Be careful to have the right format for your solution. 
	Your program will be evaluated on 10 instances of which 5 are hidden. 
	We expect you to solve all the instances.
	An unsatisfiable instance is considered as solved if the solver returns \textit{"UNSAT"}. \textbf{(2 pt)}
\end{enumerate}

\section{Propositional Logic (8 pts)}

\begin{enumerate}
	\item For each sentence, give its number of valid interpretations i.e., the number of times the sentence is true 
	(considering for each sentence {\bf all the proposition variables} $A$, $B$, $C$ and $D$). \textbf{(1 pt)}
\end{enumerate}

\begin{answers}[4cm]
	$\neg ( A \land B) \lor (\neg B \land C)$: \\
	$(\neg A \lor B) \Rightarrow C $: \\
	$( A \lor \neg B) \land (\neg B \Rightarrow \neg C) \land \neg (D \Rightarrow \neg A)$: 
\end{answers}

\newpage
\begin{enumerate}
	\setcounter{enumi}{1}
	\item Identify the goal of each set of clauses defined in \texttt{graph\_coloring.py}. \textbf{(1 pt)}
\end{enumerate}

\begin{answers}[4cm]
	Clauses \#1: \\
	Clauses \#2: \\
	Clauses \#3: \\
\end{answers}

\begin{enumerate}
	\setcounter{enumi}{2}
	\item Explain how you can express the N-amazons problem with propositional logic. For each sentence, give its meaning.
	Your model must take account of the already placed amazons. \textbf{(2 pt)}
\end{enumerate}

\begin{answers}[10cm]

\end{answers}
\newpage

\begin{enumerate}
	\setcounter{enumi}{3}
	\item Translate your model into Conjunctive Normal Form (CNF). \textbf{(2 pt)}
\end{enumerate}

\begin{answers}[10cm]

\end{answers}

\begin{enumerate}
	\setcounter{enumi}{4}
	\item Modify the function {\tt get\_expression(size)} in \texttt{amazon\_sat.py} such that it outputs a list
	of clauses modeling the n-amazons problem for the given input.
	The file \texttt{amazons\_sat.py} is the \emph{only} file that you need to modify to solve this problem. 
	Your program will be evaluated on 10 instances of which 5 are hidden. We expect you to solve all the instances.
	An unsatisfiable instance is considered as solved if the solver returns \textit{"UNSAT"}. \textbf{(2 pt)}
\end{enumerate}

\end{document}